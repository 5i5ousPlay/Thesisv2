\begin{thesisabstract}
\paragraph{         }
This study builds on the recommendations of previously done studies on semantic based location approximation as previous findings have indicated that further improvements may be made in the usage of this particular geolocation approach. In light of this, this study attempts to develop a more fine-tuned location approximation model for the geolocation of disaster-related tweets. Two semantic analysis based approaches are explored and implemented in this study: a double filter using Latent Dirichlet Allocation and Latent Semantic Analysis, and the usage of Philippine geographic representative dynamic dictionaries in multiple implementations of Latent Semantic Analysis. The accuracy of the geolocated results of both methodologies are measured against the original geolocation information of the queries processed. The results are also visualized to compare the geolocated values against the original values in a clearer manner.
\end{thesisabstract}