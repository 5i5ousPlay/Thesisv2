
\section{Context of Study}

Disaster management systems exist to provide citizens and responders a general, organized approach on how to deal with a certain situation. More systems employ the bottom-up/people-centered approach since a more authority-driven approach has been proven to be insufficient \cite{LEILIUZHANGWANYUEGUO2015, SCOLOBIGPRIORSCHRTERJRINPATT2015}. The bottom-up approach focuses on crowdsourcing, the process of gathering data based on the interactions of users within a certain spectrum.

It is important to note that lack of user participation is problematic since crowdsourcing based management systems heavily rely on data provided by external users. This means that sources of user-provided data that maximize user participation are the optimal sources of information for crowdsourcing based systems. Maximizing   user participation within a platform or network can be achieved by: (1) allowing users to post public content, (2) the existence of a trust management system, and (3) providing users with notifications for events/situations \cite{GOOLSBY2010, VICTORINOESTUARLAGMAY2016}.

Public participation, especially in times of disaster, is vital since information from various angles and contexts provide for a better idea of what is actually happening in different areas. A good source of crowdsourced information nowadays is social media sites. One example is Twitter.

Twitter is a social media networking  site that contains the features that maximize user participation. It is an excellent source of crowdsourced information since it has over a million registered users who can post content and interact with a network of other fellow users. Users may register an account using a smart phone, a tablet or a computer with internet access. Twitter allows its users to create individual profiles, post 280 character-long messages called ‘tweets’, ‘like’, repost or reply to another tweet, and connect with other users by ‘following’ them and vice versa \cite{VIEWEGSTARBIRDPALEN2010}. Content posted by users with public profiles is openly accessible by anyone on the internet. Twitter has been used as a tool to circulate information regarding emergency and disaster related incidents \cite{GOOLSBY2010, VIEWEGSTARBIRDPALEN2010, MCDOUGALL2011} .   Incidents such as the terrorist attacks in Mumbai, India in 2008 and the Queensland floods in Australia in 2011 have shown how social media sites, Twitter in particular, contribute to better disaster risk management and response.

A key component in Twitter that contributes to disaster management is the geolocation information that users may provide. Users provide latitude and longitude coordinates either by indicating a location on their profiles or by turning on their location feature whenever they tweet. Being able to track where a disaster related tweet was sent from gives that tweet more depth as it describes something that is happening at a particular location. The more accurate information on a certain location, the easier it will be for responders to decide how to handle the situation in that location.

However, as large a database of useful information Twitter is, only a small percentage of tweets are geotagged. Previous research found that only around 0.42-1 percent of users allow their tweets to be geotagged \cite{CHENGCAVARLEELEE2010, MAHMUDNICHOLSDREWS2014} and less than 2 percent of total tweets contain location details \cite{LAYLAVIRAJABIFARDKALANTARI2016}. While using Twitter data is effective in disaster management, methods on how to process these data can still be improved.

This opens the door for location approximation for disaster management. Since the majority of tweets are non-geotagged, the need for methods on how to approximate the location of a non-geotagged tweet using geotagged tweets becomes clear. Previous studies have explored different methods of approximating the location of non-geotagged tweets \cite{OBCS2013, GFC2012,carmen,ROSALES2017, VELASCOBERMEJODOMINGO2018}. These previous methods of location approximation include text analysis \cite{ROSALES2017, VELASCOBERMEJODOMINGO2018}, geocoding of user information \cite{OBCS2013, GFC2012, carmen}, and reverse geocoding of geographical coordinates \cite{ROSALES2017, VELASCOBERMEJODOMINGO2018}. Results have shown that there is still room for improvement regarding previously explored location approximation methods--text analysis methods proved to be not completely accurate due to the ungrammatical characteristics of tweets \cite{ROSALES2017, VELASCOBERMEJODOMINGO2018}, geocoding of user information proved to be difficult since users may lie about their "home" location \cite{OBCS2013, GFC2012}, and reverse geocoding of geographical coordinates do not always yield valid map locations.   

Thus, the goal of this study is to derive and test Latent Semantic Analysis based methods aimed towards the fine-tuning of previously used concepts and algorithms. Different applications of existing location approximation methods are being explored in order to determine which direction leads to more accurate geolocation results. Algorithms such as Latent Semantic Analysis and Latent Dirichlet Allocation are being used to test the different methods in this study. The results are to be formatted as a JSON file that can be available for graphical visualization